% Programming Languages (H) CW Report written in LaTeX by Inesh Bose

\documentclass{article}


%==============================================================================
%% Packages and Command Setting
\usepackage[utf8]{inputenc}
\setcounter{secnumdepth}{0}
\usepackage[shortlabels]{enumitem}
\usepackage{layout}
\usepackage{multirow}
\usepackage{hyperref}
\usepackage{minted}
\usepackage{caption}
\usemintedstyle{friendly}
\newcommand{\code}[1]{\texttt{#1}}
\newenvironment{codelst}{\captionsetup{type=listing}}{}


%=====================================
%% Options to Show/Hide Sections
%% Since page limit is 4, this is advantageous
%% Content formatting has been kept in mind

\newcommand{\showtoc}{0}
\newcommand{\showinstructions}{1}
\newcommand{\showdetails}{1}
\newcommand{\dividestages}{0}
\newcommand{\showcontextcode}{1}
\newcommand{\showtemplate}{0}
\newcommand{\showobjectcode}{0}
\newcommand{\showencodecode}{1}


%=====================================
%% Extra Commands for Customisation

\newcommand{\npage}{
    \if\dividestages1
        \newpage
    \fi
}
\newcommand{\stage}[1]{
    \if\dividestages1
        \vspace{0.5cm}
        \subsection{#1}
    \fi
}
\newcommand{\act}[1]{
    \if\dividestages1
        \subsubsection{#1}
    \fi
}
\newcommand{\codecaption}[1]{
    \if\dividestages0
        \vspace{-0.75cm}
        \caption{\textbf{#1}}
        \vspace{0.4cm}
    \fi
}


%==============================================================================
%% Title Page

\title{Programming Languages (H)\\Coursework Status Report}
\author{\bf Inesh Bose}
\date{}

\begin{document}

\definecolor{bg}{rgb}{0.95,0.95,0.95}

\maketitle

\if\showtoc1

    \vspace{2cm}

    \if\dividestages0
        \vspace{2cm}
    \fi

    \tableofcontents
    \addtocontents{toc}{\protect\thispagestyle{empty}}

\fi

\pagenumbering{gobble}


%==============================================================================
%% Status

\if\showtoc1

\newpage

\begin{codelst}
\begin{minted}[bgcolor=bg]{text}
//////////////////////////////////////////////////////////////
//
// Fun Compiler
//
// Based on a previous version developed by
// David Watt and Simon Gay (University of Glasgow).
//
// Extended by Inesh Bose
//
//////////////////////////////////////////////////////////////
\end{minted}
\end{codelst}

\fi

\vspace{0.5cm}

\section{Status}

\noindent The coursework has been completed and tested successfully. The Fun compiler has been extended with a \code{for} command and a \code{switch} command. The directory organisation is maintained, and the code is strictly in the same format as the rest of the files. A few personal programming style preferences are reflected such naming variables and avoiding excessive commenting.

\if\showinstructions1

\vspace{0.5cm}

\if\showtoc1
\vspace{0.5cm}
\fi

\begin{codelst}
\begin{minted}[tabsize=5,bgcolor=bg,fontsize=\footnotesize]{bash}
# Generate lexer with defined grammar
> java -jar antlr.jar -no-listener -visitor -o src/ast src/ast/Fun.g4

# Compile FunCheck for Contextual Analysis
> javac -cp "antlr.jar" -d bin/ -sourcepath src/ src/fun/FunCheck.java
> java -cp "antlr.jar;bin" fun/FunCheck tests/switch1.fun

# Compile FunRun for Code Generation
> javac -cp "antlr.jar" -d bin/ -sourcepath src/ src/fun/SVM.java
> javac -cp "antlr.jar" -d bin/ -sourcepath src/ src/fun/FunRun.java
> java -cp "antlr.jar;bin" fun/FunRun tests/switch1.fun
\end{minted}
\codecaption{How-to-Run Instructions}
\end{codelst}

\fi

\vspace{0.5cm}

\noindent The \code{zip} file includes all files (including unchanged) and \code{antlr.jar} in the \code{FunExtended/} directory that is in the same level as this report \code{StatusReport}. Be sure to \code{cd} into the directory.


%==============================================================================
%% For Command

\npage

\section{\textit{\code{for} command}}

\noindent The \code{for} command was easily implemented and did not require \textit{"out-of-the-way"} changes. The syntax was defined, and visitors were simply overridden.

\if\showdetails1

%=====================================
%% For Command : Syntactical Analysis

\stage{Syntactical Analysis}

\noindent To define the grammar for the \code{for} command, the \code{com} token (that defines a command) was extended. Only \code{FOR} and \code{TO} were added to the lexicons.

\begin{codelst}
\begin{minted}[tabsize=5,bgcolor=bg]{text}
com
    :   ....
    |   FOR ID ASSN e1=expr TO e2=expr
          COLON seq_com DOT		  # for
    |   ....
    ;
\end{minted}
\codecaption{Syntax defined in \code{Fun.g4}}
\end{codelst}


%=====================================
%% For Command : Contextual Analysis

\stage{Contextual Analysis}

\noindent The scope/type rules for a \code{for} command requires the \code{ID} to be already declared and of type \code{int} (so \code{e1} and \code{e2} are also \code{int}).

\if\showcontextcode1

\begin{codelst}
\begin{minted}[tabsize=5,bgcolor=bg]{java}
public Type visitFor(FunParser.ForContext ctx) {
	checkType(Type.INT, retrieve(ctx.ID().getText(), ctx), ctx);
	Type t1 = visit(ctx.e1);
	Type t2 = visit(ctx.e2);
	visit(ctx.seq_com());
	checkType(Type.INT, t1, ctx);
	checkType(Type.INT, t2, ctx);
	return null;
}
\end{minted}
\codecaption{\code{FunCheckerVisitor.java}}
\end{codelst}

\npage

\fi

%=====================================
%% For Command : Code Generation

\stage{Code Generation}

\noindent A \textit{for-loop} is similar to a \textit{while-loop}, therefore \code{visitWhile} was used as reference. The difference is that \code{ID} will be assigned a value at the start and is incremented at the end using \code{SVM.INC}.

\act{Code Template}

\vspace{0.5cm}

\noindent The template shares how the first \code{expr} is evaluated and stored into \code{ID}, then noting address before the second \code{expr} to evaluate condition.

\if\showtemplate1

\vspace{0.5cm}

\begin{codelst}
\begin{minted}[escapeinside=||,tabsize=5,bgcolor=bg]{text}
1. walk |\textit{expr}|, generating code;
2. lookup |\textbf{ID}| and retrieve its address |\textit{d}|;
3. emit instruction "STOREG/STOREL |\textit{d}|" depending on scope;
4. note the current instruction address |\textit{c1}|;
5. walk |\textit{expr}|, generating code;
6. emit "JUMPT 0" (after CMPGT);
7. walk |\textit{com}|, generating code;
8. increment stored value at |\textit{d}|;
9. emit "JUMP c1";
\end{minted}
\codecaption{Code Template}
\end{codelst}

\if\showobjectcode1

\noindent The object code then shows a set of corresponding instructions.

\begin{codelst}
\begin{minted}[tabsize=5,bgcolor=bg]{text}
 7: LOADC   e1
10: STOREG  ID
13: LOADG   ID
16: LOADC   e2
19: CMPLT
20: JUMPF   39
23: LOADG   ID
26: CALL    40
29: LOADG   ID
32: INC
33: STOREG  ID
36: JUMP    13
39: RETURN  0
\end{minted}
\codecaption{Object Code}
\end{codelst}

\npage

\fi

\fi

\act{Encoder}

\vspace{0.5cm}

\noindent The code generator / encoder then uses this idea and implements it in \code{visitFor}. After getting the address and scope using \code{locale} of the \code{ID}, the \code{expr}s are evaluated and jumps are handled accordingly.

\if\showencodecode1

\vspace{0.5cm}

\begin{codelst}
\begin{minted}[tabsize=5,bgcolor=bg]{java}
visit(ctx.e1);
obj.emit12(store, varaddr.offset);
int startaddr = obj.currentOffset();
obj.emit12(load, varaddr.offset);
visit(ctx.e2);
obj.emit1(SVM.CMPGT);
int condaddr = obj.currentOffset();
obj.emit12(SVM.JUMPT, 0);
visit(ctx.seq_com());
obj.emit12(load, varaddr.offset);
obj.emit1(SVM.INC);
obj.emit12(store, varaddr.offset);
obj.emit12(SVM.JUMP, startaddr);
obj.patch12(condaddr, obj.currentOffset());
\end{minted}
\codecaption{\code{FunEncoderVisitor.java}}
\end{codelst}

\fi

%=====================================
%% For Command : Tests

\stage{Tests}

\noindent \code{for1.fun} includes a very simple \textit{for-loop} inside \code{main} that is expected to write the value of \code{i} (\code{2 to 5}) at every loop. \code{for2.fun} expects to print factorial value (\code{of 5}) using the \textit{for-loop}.

\fi


%==============================================================================


\if\showdetails0
\vspace{4cm}
\fi


%==============================================================================
%% Switch Command

\vspace{0.5cm}

\if\showobjectcode1
\npage
\fi

\section{\textit{\code{switch} command}}

\noindent Extending the Fun compiler with the \code{switch} command required more changes. The required lexicons were \code{SWITCH}, \code{CASE} and \code{DEFAULT}. An early approach to define the syntax was to put it all under one Backus-Naur Form expression.

\if\showdetails1

%=====================================
%% Switch Command : Syntactical Analysis

\stage{Syntactical Analysis}

\begin{codelst}
\begin{minted}[tabsize=5,bgcolor=bg]{text}
com
    :   ....
    |   SWITCH ID COLON
          ( CASE ( NUM
                 | ( FALSE | TRUE )
                 | (n1=NUM DOT DOT n2=NUM)
                 )
            COLON seq_com
          )*
          DEFAULT COLON seq_com
          DOT					  # switch
    |   ....
    ;
\end{minted}
\codecaption{Early version of syntax defined in \code{Fun.g4}}
\end{codelst}

\vspace{0.2cm}

\noindent This, however, caused inconvenience and unnecessary complexities during the later stages. Hence, the grammar was broken down into \code{scase} (standard case), \code{dcase} (default case) and \code{guard} tokens.


\begin{codelst}
\begin{minted}[tabsize=5,bgcolor=bg]{text}
com
    :   ....
    |   SWITCH expr COLON
          scase*
          dcase DOT			      # switch
    |   ....
    ;

scase
    :   CASE guard
          COLON seq_com
    ;

dcase
    :   DEFAULT COLON seq_com
    ;

guard
    :   ( NUM
        | ( FALSE | TRUE )
        | (n1=NUM DOT DOT n2=NUM)
        )
    ;
\end{minted}
\codecaption{Syntax broken into separate tokens in \code{Fun.g4}}
\end{codelst}

%=====================================
%% Switch Command : Contextual Analysis

\stage{Contextual Analysis}

\noindent Majority of contextual analysis happened in \code{visitSwitch}. The \code{expr} had to be of the same type as the guards, and the guards had to be unique and not overlapping (in the case of ranges); this was done by storing the guards as strings in a set. This set was populated with a temporary set created for each guard; this way, a range could be represented as a set of numbers going from \code{n1} to \code{n2}. So now, if there was an intersect between the set was being added and the set of all guards, there will be an error. This appeared to be a clean and efficient way to check duplicates/overlap two-way.

\vspace{0.5cm}

\begin{codelst}
% Shown irrespective of \showcontextcode as this is important and good to show
\begin{minted}[tabsize=5,bgcolor=bg]{java}
if (Collections.disjoint(guards, gtemp)) {
	guards.addAll(gtemp);
}
else {
	reportError("Guard duplicated or overlapped!", ctx);
}
\end{minted}
\codecaption{\code{FunCheckerVisitor.java}}
\end{codelst}

%=====================================
%% Switch Command : Code Generation

\stage{Code Generation}

\noindent A \code{switch-command} can be seen similar to \code{if-statements} that only work with \code{==}. The catch is that unlike \code{if-else}, the number of cases is unknown.

\vspace{1cm}

\act{Code Template}

\noindent The template would repeat itself while walking over every case hence the template is structured accordingly and instructions are written as sub-lists.

\if\showtemplate1

\vspace{0.5cm}

\begin{codelst}
\begin{minted}[escapeinside=||,tabsize=5,bgcolor=bg]{text}
1. for every case,
	1. walk |\textit{expr}|, generating code;
	2. walk |\textit{guard}|, generating code;
	3. note the current instruction address |\textit{c1}|;
	4. emit "JUMPF 0" (after CMPEQ/CMPIN);
	5. walk |\textit{scase}| (|\textit{com}|), generating code;
	6. note the current instruction address |\textit{c2}|
	    in list of case addresses;
	7. emit "JUMP 0";
	8. note the current instruction address |\textit{c3}|;
	9. patch |\textit{c1}| to |\textit{c3}| for next case jump address;
2. walk |\textit{dcase}| (|\textit{com}|), generating code;
3. note the current instruction address |\textit{c4}|;
4. for every address in list of case addresses,
	1. patch address |\textit{c2}| to |\textit{c4}| for
	    exiting / breaking out switch context;
\end{minted}
\codecaption{Code Template}
\end{codelst}

\if\showobjectcode1

\vspace{0.5cm}

\noindent The generated object code displays how the jumps for each case statement is placed along with their commands (\code{seq\_com}).

\vspace{0.5cm}

\begin{codelst}
\begin{minted}[tabsize=5,bgcolor=bg]{text}
 4: LOADC   3
 7: LOADL   2
10: LOADC   1
13: CMPEQ
14: JUMPF   26
17: LOADC   9
20: STOREL  2
23: JUMP    54
26: LOADL   2
29: LOADC   2
32: LOADC   4
35: CMPIN
36: JUMPF   48
39: LOADC   5
42: STOREL  2
45: JUMP    54
48: LOADC   1
51: STOREL  2
\end{minted}
\codecaption{Object Code}
\end{codelst}

\fi

\fi


\act{\code{SVM.java}}

\vspace{0.5cm}

\noindent An instruction had to be added to the SVM in order to check "\textit{equality}" with a range guard. This was done by adding \code{CMPIN} instruction.

\vspace{0.5cm}

\begin{codelst}
\begin{minted}[tabsize=5,bgcolor=bg]{java}
case CMPIN: {
	int w3 = data[--sp];
	int w2 = data[--sp];
	int w1 = data[--sp];
	Set<Integer> range = new HashSet<Integer>();
	for (int i=w2; i<=w3; i++)
		range.add(i);
	data[sp++] = (range.contains(w1) ? 1 : 0);
	break;
}
\end{minted}
\codecaption{\code{SVM.java}}
\end{codelst}

\noindent This was useful in \code{visitGuard} that would load values to the register, and do an operation accordingly. (\code{CMPIN} for ranges, and \code{CMPEQ} for others).

\vspace{0.5cm}

\act{Encoder}

\vspace{0.5cm}

\noindent The \code{expr} and \code{guard} would have to be added in the register every time in order to do an operation. Following that, the instructions are similar to an \code{if} statement involving \code{JUMPF} and noting few addresses. These addresses have to be patched for proper ordering.

\if\showencodecode1

\vspace{0.5cm}

\begin{codelst}
\begin{minted}[tabsize=5,bgcolor=bg]{java}
visit(ctx.expr());
visit(c.guard());
int jumpaddr = obj.currentOffset();
obj.emit12(SVM.JUMPF, 0);
visit(c);
int endaddr = obj.currentOffset();
casesaddr.add(endaddr);
obj.emit12(SVM.JUMP, 0);
int nextaddr = obj.currentOffset();
obj.patch12(jumpaddr, nextaddr);
\end{minted}
\codecaption{\code{visitSwitch} in \code{FunEncoderVisitor.java}}
\end{codelst}

\vspace{0.5cm}

\fi

\vspace{0.5cm}

\noindent On visiting the guard (\code{visitGuard}), the essential values are added to the register and appropriate operation is emitted.

\vspace{0.75cm}

\begin{codelst}
\begin{minted}[tabsize=2,bgcolor=bg]{java}
if (ctx.DOT().size() > 0) {
	obj.emit12(SVM.LOADC, Integer.parseInt(ctx.n1.getText()));
	obj.emit12(SVM.LOADC, Integer.parseInt(ctx.n2.getText()));
	obj.emit1(SVM.CMPIN);
}
else if (ctx.NUM().size() > 0) {
	obj.emit12(SVM.LOADC, Integer.parseInt(ctx.NUM(0).getText()));
	obj.emit1(SVM.CMPEQ);
}
else {
	obj.emit12(SVM.LOADC, (ctx.TRUE() != null) ? 1 : 0);
	obj.emit1(SVM.CMPEQ);
}
\end{minted}
\codecaption{\code{visitGuard} in \code{FunEncoderVisitor.java}}
\end{codelst}

%=====================================
%% Switch Command : Tests

\stage{Tests}

\vspace{0.5cm}

\noindent Finally, the \code{switch} command was tested with a simple Boolean switch command in \code{main} (\code{switchsimple.fun}), an integer switch with no range (\code{switch1.fun}), one similar but with a range (\code{switch2.fun}), with case corresponding to default (\code{switch3.fun}), one integer and one Boolean switch command in functions being called (\code{switch4.fun}) provided from the assignment sheet, one with more than one range and overlap that should not compile (\code{switch5.fun}) and one with a nested switch (\code{switchswitch.fun}).

\fi


%==============================================================================

\end{document}
